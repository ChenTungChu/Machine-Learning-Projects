
\documentclass[11pt]{article}

\usepackage{fullpage}
\usepackage{amsmath,amssymb,amsthm,amsfonts,latexsym,bbm,xspace,graphicx,float,mathtools,
verbatim, xcolor} 
\PassOptionsToPackage{hyphens}{url}\usepackage{hyperref}
\newcommand{\new}[1]{\textcolor{red}{#1}}
%\usepackage{psfig}

\newcommand{\future}[1]{\textcolor{red}{#1}}

\newcommand{\hP}{\hat P}
\newcommand{\hp}{\hat p}

\newcommand{\Dk}{\Delta_k}
\newcommand{\Px}{P(x)}
\newcommand{\Qx}{Q(x)}
\newcommand{\Nx}{N_x}

\newcommand{\Py}{P(y)}
\newcommand{\Qy}{Q(y)}
\newcommand{\Pml}{P_{ML}}
\newcommand{\Pmlx}{\Pml(x)}
\newcommand{\Pbeta}{P_{\beta}}
\newcommand{\Pbetax}{\Pbeta(x)}


\newcommand{\dTV}[2]{d_{TV} (#1,#2)}
\newcommand{\dKL}[2]{D(#1||#2)}
\newcommand{\chisq}[2]{\chi^2(#1,#2)}
\newcommand{\eps}{\varepsilon}

\newcommand{\nPepsp}[1]{n^*(#1, \eps)}
\newcommand{\nPeps}{\nPepsp{\cP}}


\newcommand{\sumX}{\sum_{x\in\cX}}

\newcommand{\Bpr}[1]{Bern(#1)}

\newenvironment{problem}[2][Problem]{\begin{trivlist}
\item[\hskip \labelsep {\bfseries #1}\hskip \labelsep {\bfseries #2.}]}{\end{trivlist}}

% Theorem-like environments

\newtheorem{Theorem}{Theorem}
\newtheorem{Theorem*}{Theorem}

\newtheorem{Claim}[Theorem]{Claim}
\newtheorem{Claim*}[Theorem]{Claim}
\newtheorem{Corollary}[Theorem]{Corollary}
\newtheorem{Conjecture}[Theorem]{Conjecture}
\newtheorem{CounterExample*}{$\overline{\hbox{\bf Example}}$}
\newtheorem{Definition}[Theorem]{Definition}
\newtheorem{Example}[Theorem]{Example}
\newtheorem{Example*}[Theorem]{Example}
\newtheorem{Exercise}[Theorem]{Exercise}
\newtheorem{Intuition*}[Theorem]{Intuition}
\newtheorem{Joke*}[Theorem]{Joke}
\newtheorem{Lemma}[Theorem]{Lemma}
\newtheorem{Lemma*}[Theorem]{Lemma}
\newtheorem{Open problem}[Theorem]{Open problem}
\newtheorem{Proposition}[Theorem]{Proposition}
\newtheorem{Property}[Theorem]{Property}
\newtheorem{Question}[Theorem]{Question}
\newtheorem{Question*}[Theorem]{Question}
\newtheorem{Remark}[Theorem]{Remark}
\newtheorem{Result}[Theorem]{Result}
\newtheorem{Fact}[Theorem]{Fact}
\newtheorem{Condition}[Theorem]{Condition}

\newcommand{\ed}{\stackrel{\mathrm{def}}{=}}
%\newcommand{\edef}{\stackrel{\mathrm{def}}{=}}
\def \Paren#1{{\left({#1}\right)}}

\newcommand{\probof}[1]{\Pr\Paren{#1}}
\newcommand{\proboff}[1]{p\Paren{#1}}


\newtheorem{theorem}{Theorem}
\newtheorem{proposition}[theorem]{Proposition}
\newtheorem{corollary}[theorem]{Corollary}
%\newtheorem*{corollary*}{Corollary}
\newtheorem{assumption}[theorem]{Assumption}
\newtheorem{lemma}[theorem]{Lemma}
%\newtheorem*{lemma*}{Lemma}
\newtheorem{conjecture}[theorem]{Conjecture}
\newtheorem{example}[theorem]{Example}
\newtheorem{definition}[theorem]{Definition}
\newtheorem{claim}[theorem]{Claim}

\newcommand{\Xon}{X_1^n}


\newcommand{\prob}{{\rm Pr}}
\newcommand{\Probof}[1]{\prob\left(#1\right)}



\newcommand{\ignore}[1]{}

% Equation formatting

\newcommand{\spreqn}[1]{{\qquad\text{#1}\qquad}}

% Blackboard fonts
\newcommand{\II}{\mathbb{I}} % Added by Theertha on April 16th 2013.
\newcommand{\EE}{\mathbb{E}}
\newcommand{\CC}{\mathbb{C}}
\newcommand{\NN}{\mathbb{N}}
\newcommand{\QQ}{\mathbb{Q}}
\newcommand{\RR}{\mathbb{R}}
\newcommand{\ZZ}{\mathbb{Z}}
\newcommand{\PP}{\mathbb{P}}

% Number sets

\newcommand{\complex}{\CC}
\newcommand{\integers}{\ZZ}
\newcommand{\naturals}{\NN}
\newcommand{\positives}{\PP}
\newcommand{\rationals}{\QQ}
\newcommand{\reals}{\RR}

\newcommand{\realsge}{{\reals_{\ge}}}
\newcommand{\realsp}{\reals^+}
\newcommand{\integersp}{\integers^+}
\newcommand{\integerss}[1]{\integers_{\ge{#1}}}

% boldface

\def \ba     {{\bf a}}
\def \bx     {{\bf x}}
\def \by     {{\bf y}}

\def \bA     {{\bf A}}
\def \bB     {{\bf B}}
\def \bC     {{\bf C}}
\def \bD     {{\bf D}}
\def \bF     {{\bf F}}
\def \bG     {{\bf G}}
\def \bL     {{\bf L}}
\def \bQ     {{\bf Q}}
\def \bR     {{\bf R}}
\def \bS     {{\bf S}}
\def \bT     {{\bf T}}
\def \bX     {{\bf X}}
\def \bY     {{\bf Y}}
\def \bZ     {{\bf Z}}

% caligraphics

\def \cA     {{\cal A}}
\def \cB     {{\cal B}}
\def \cC     {{\cal C}}
\def \cD     {{\cal D}}
\def \cE     {{\cal E}}
\def \cF     {{\cal F}}
\def \cG     {{\cal G}}
\def \cH     {{\cal H}}
\def \cI     {{\cal I}}
\def \cK     {{\cal K}}
\def \cL     {{\cal L}}
\def \cM     {{\cal M}}
\def \cN     {{\cal N}}
\def \cO     {{\cal O}}
\def \cP     {{\cal P}}
\def \cQ     {{\cal Q}}
\def \cR     {{\cal R}}
\def \cS     {{\cal S}}
\def \cT     {{\cal T}}
\def \cU     {{\cal U}}
\def \cV     {{\cal V}}
\def \cW     {{\cal W}}
\def \cX     {{\cal X}}
\def \cY     {{\cal Y}}
\def \cZ     {{\cal Z}}

% vectors

\def \vct#1{{\overline{#1}}}

\def \vcta  {{\vct a}}
\def \vctb  {{\vct b}}
\def \vctq  {{\vct q}}
\def \vcts  {{\vct s}}
\def \vctu  {{\vct u}}
\def \vctv  {{\vct v}}
\def \vctx  {{\vct x}}
\def \vcty  {{\vct y}}
\def \vctz  {{\vct z}}
\def \vctp  {{\vct p}}

\def \vctV  {{\vct V}}
\def \vctX  {{\vct X}}
\def \vctY  {{\vct Y}}
\def \vctZ  {{\vct Z}}

\def \vctbeta  {{\vct\beta}}



% qed's --  Also consider \qedhere

\def \eqed    {\eqno{\qed}}
\def \rqed    {\hbox{}~\hfill~$\qed$}

% sequences

\def \upto  {{,}\ldots{,}}

\def \zn    {0\upto n}
\def \znmo  {0\upto n-1}
\def \znpo  {0\upto n+1}
\def \ztnmo {0\upto 2^n-1}
\def \ok    {1\upto k}
\def \on    {1\upto n}
\def \onmo  {1\upto n-1}
\def \onpo  {1\upto n+1}

% sets

\def \sets#1{{\{#1\}}}
\def \Sets#1{{\left\{#1\right\}}}

\def \set#1#2{{\sets{{#1}\upto{#2}}}}

\def \setpmo   {\sets{\pm 1}}
\def \setmpo   {\{-1{,}1\}}
\def \setzo    {\{0{,}1\}}
\def \setzn    {\{\zn\}}
\def \setznmo  {\{\znmo\}}
\def \setztnmo {\{\ztnmo\}}
\def \setok    {\{\ok\}}
\def \seton    {\{\on\}}
\def \setonmo  {\{\onmo\}}
\def \setzon   {\setzo^n}
\def \setzos   {\setzo^*}


% Set operations



\def\ignore#1{}

 

\title{Assignment Four\\ ECE 4200/5420}

\begin{document}
\maketitle 

\begin{itemize}
\item
Provide credit to \textbf{any sources} other than the course staff that helped you solve the problems. This includes \textbf{all students} you talked to regarding the problems.\item
		Questions marked with an asterisk \textbf{is optional for students taking the class as ECE 4200}. 	
\item
You can look up definitions/basics online (e.g., wikipedia, stack-exchange, etc).
\item
{\bf The due date is 10/08/2021, 23.59.59 ET}. 
\item
Submission rules are the same as previous assignments.

\end{itemize}



\noindent In \textbf{Problems 1, 2, and 3} we will study linear regression. We will assume in these problems that $w^0=0$. (This can be done by centering the features and labels to have mean 0, but we will not worry about it).  For $\bar w = (w^1, \ldots, w^d)$, and $ X = ( X^1, \ldots,  X^d)$, the regression we want is:
\begin{align}
y = w^1 \bar X^1+\ldots + w^d \bar X^d = \bar w\cdot  X.
\end{align}
We considered the following regularized least squares objective, which is called as \textbf{Ridge Regression}. For the dataset $S=\{( X_1, y_1), \ldots, ( X_n, y_n)\}$,  
\begin{align}
J(\bar w, \lambda) = \sum_{i=1}^n \Paren{y_i -\bar w\cdot X_i}^2 +\lambda \cdot \|\bar w\|_2^2.\label{eqn:ridge}
\end{align}
We then find a $\bar w$ to minimize $J(\bar w, \lambda)$, namely
\begin{equation}
\arg\min_{\bar w} J(\bar w, \lambda).
\label{eqn:optim}
\end{equation}
\begin{problem}{1 (10 points) Gradient Descent for regression}\quad

\begin{enumerate}
\item 
Instead of using the closed form expression we mentioned in class, suppose we want to perform gradient descent to find the optimal solution for $J(\bar w)$. Please compute the gradient of $J$, and write one step of the gradient descent with step size $\eta$. 
\item
Suppose we get a new point $ X_{n+1}$, what will the predicted $y_{n+1}$ be when $\lambda\to\infty$? 
\end{enumerate}
\end{problem}


\begin{problem}{2 (15 points) Regularization increases training error}
Note the two terms in $J(\bar w, \lambda)$. The first is the training error, and the second is the regularization. When $\lambda=0$ then we minimize just the training error. As $\lambda$ increases, the second term will become more and more dominant, and this means two things:  (1) We are giving more weight to the regularization term and therefore, the training error should perhaps decrease, and (2) and as $\lambda$ increases we should obtain a $\bar w$ with smaller norm. We will formalize both these things rigorously below. 

Let $0<\lambda_1<\lambda_2$ be two regularizer values. Let $\bar w_1$, and $\bar w_2$ be the minimizers of $J(\bar w, \lambda_1)$, and $J(\bar w, \lambda_2)$ respectively. 

\begin{enumerate}
	\item Show that $\|\bar w_1\|_2^2\ge \|\bar w_2\|_2^2$. Therefore more regularization implies smaller norm of solution!
	
	\textbf{Hint:} Observe that $J(\bar w_1, \lambda_1)\le J(\bar w_2, \lambda_1)$, and $J(\bar w_2, \lambda_2)\le J(\bar w_1, \lambda_2)$ (why?). 
	\item Show that the training error for $\bar w_1$ is less than that of $\bar w_2$. In other words, show that 
\[
\sum_{i=1}^n \Paren{y_i -\bar w_1\cdot \bar X_i}^2 \le \sum_{i=1}^n \Paren{y_i -\bar w_2\cdot \bar X_i}^2.
\]
This shows that as we regularize more, the training error grows.
\textbf{Hint:} Use the first part of the problem.
\end{enumerate}

\end{problem}

%\begin{problem}{3 (15 points)}
%Suppose we have a classification problem, where each example $\bar X$ has $k$ \textbf{positive} features. Suppose we restrict weight vectors to be \textbf{positive}, and consider the following discriminative function:
%\[
%Pr (y= -1 | \bar X, \bar w, t ) = \frac{2}{1+\exp(\bar w\cdot \bar X-t)}.
%\]
%The positivity means that each entry of $\bar w$ is positive, and $t$ is also positive.
%\begin{enumerate}
%\item
%Find the gradient of $(\bar X\cdot \bar w)^2$ with respect to $\bar w, t$.
%\item 
%Write the log-likelihood function $\ell(\bar w, t)$ for this problem.
%\item
%Is $\ell(\bar w, t)$ concave in the positive quadrant? 
%\item
%Compute the gradient of $\ell(\bar w)$.
%\item
%Write one step of the gradient ascent step to find the maximum likelihood estimate of $\bar w$.
%\end{enumerate}
%\end{problem}

%Recall ridge regression with features in $\RR^d$. The objective is to find $\bar w = (\bar w^0, \bar w^1, \ldots, \bar w^d)$ and then for a feature $\bar X\in\RR^d$, we predict $y= \bar w^0+ \bar w^1  X^1+\ldots + \bar w^d  X^d=\tilde w\cdot \tilde X$ where $\tilde X = (1,  X^1, \ldots,  X^d)$. In least squares regression with a regularization $\lambda$ (ridge regression), we considered the following objective function:
%\begin{align}
%J(\bar w, \lambda)=\sum_{i=1}^n \Paren{y_i -\tilde w\cdot \tilde X_i}^2 +{\lambda} \cdot \|\bar w\|_2^2.
%\label{eqn:ridge}
%\end{align}
%We then find a $\bar w$ to minimize $J(\bar w, \lambda)$, namely
%\begin{equation}
%\arg\min_{\bar w} J(\bar w, \lambda).
%\label{eqn:optim}
%\end{equation}

\begin{problem}{3$^*$ (15 points) Ridge regression $\equiv$ MAP estimation with Gaussian prior}

In  class we provided a Maximum Likelihood (ML) interpretation of least square regression without regularization (i.e., $\lambda=0$) under the Gaussian noise model. 

The Gaussian noise model is $y_i= \bar w\cdot \bar X_i +N(0,\sigma^2)$, where $N(0,\sigma^2)$ is a Gaussian distribution with mean 0 and variance $\sigma^2$. In other words, 
\begin{align}
\proboff{y_i-\bar w\cdot \bar X_i=\nu|\bar w, \bar X} = \frac1{\sqrt{2\pi\sigma^2}}\exp\left(-\frac{\nu^2}{2\sigma^2}\right). 
\label{eqn:gaussian-model}
\end{align}

 We proved in class that Equation~\eqref{eqn:ridge} with $\lambda=0$  is equivalent to Maximum Likelihood estimation under the Gaussian noise model. 
 
The goal of this problem is to show that least squares regression with regularization is equivalent to Maximum Aposteriori (MAP) estimation under a Gaussian prior over $\bar w$. 

Suppose the noise model is still Gaussian (Equation~\eqref{eqn:gaussian-model}). Furthermore, suppose that $\bar w$ has a \textbf{prior distribution} that is Gaussian with mean $0$ and variance $\sigma^2/\lambda$, namely
\[
\proboff{\bar w}=\frac1{(2\pi\sigma^2/\lambda)^{d/2}}\exp\left(-\frac{\lambda\|\bar w\|_2^2}{2\sigma^2}\right).
\]

\begin{enumerate}
\item 
Show that the MAP estimator under Gaussian noise model and Gaussian prior:
\begin{align}
\label{eqn:linear}
\arg\max_{\bar w} \proboff{w|y_1,\ldots, y_n,  X_1,\ldots,  X_n}	
\end{align}
is equivalent to solving~\eqref{eqn:optim}. 

\textbf{Hint:} Start by noting that
\[
\arg\max_{\bar w}\proboff{\bar w|y_1,\ldots, y_n,  X_1,\ldots,  X_n} = \arg\max_{\bar w}\proboff{y_1,\ldots, y_n|  X_1,\ldots,  X_n, \bar w} \cdot \proboff{\bar w}.
\]
Now the first term is what we had in class for ML interpretation, and the second term is what will introduce the regularization.
\item
 What is the MAP estimator $\bar w$ as $\lambda\to\infty$, namely when the prior distribution of $\bar w$ is Gaussian with mean 0 and variance close to 0?
\vfill
\end{enumerate}
\end{problem}




\begin{problem}{4 (25 points) Linear and Quadratic Regression}
Please refer to the Jupyter Notebook in the assignment, and complete the coding part in it!
You can use sklearn regression package: \url{http://scikit-learn.org/stable/modules/generated/sklearn.linear_model.Ridge.html}
\end{problem}





\end{document}

If the model were trained with wine training2.txt, which λ would you choose to train your final model? Why?

%
%\begin{table*}[h]
%\begin{center}
%\begin{tabular}{c|ccccc}
%{\bf Liked?} &
%{\bf Easy?} &
%{\bf AI?} &
%{\bf Sys?} &
%{\bf Thy?} &
%{\bf Morning?} \\
%\hline
%+ & y  & y  & n  & y  & n  \\
%+ & y  & y  & n  & y  & n  \\
%+ & n  & y  & n  & n  & n  \\
%+ & n  & n  & n  & y  & n  \\
%+ & n  & y  & y  & n  & y  \\
%+ & y  & y  & n  & n  & n  \\
%+ & y  & y  & n  & y  & n  \\
%+ & n  & y  & n  & y  & n  \\
%+ & n  & n  & n  & n  & y  \\
% + & y  & n  & n  & y  & y  \\
% + & n  & y  & n  & y  & n  \\
% + & y  & y  & y  & y  & y  \\
%- & y  & y  & y  & n  & y  \\
%- & n  & n  & y  & y  & n  \\
%- & n  & n  & y  & n  & y  \\
%- & y  & n  & y  & n  & y  \\
%- & n  & n  & y  & y  & n  \\
%- & n  & y  & y  & n  & y  \\
%- & y  & n  & y  & n  & n  \\
%- & y  & n  & y  & n  & y  \\
%\end{tabular}
%\end{center}
%\caption{Course rating data set}
%\label{tab:data:course}
%\end{table*}
