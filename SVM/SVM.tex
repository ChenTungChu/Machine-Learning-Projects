
\documentclass[11pt]{article}

\usepackage{fullpage}
\usepackage{amsmath,amssymb,amsthm,amsfonts,latexsym,bbm,xspace,graphicx,float,mathtools,
verbatim, xcolor,bm} 
\PassOptionsToPackage{hyphens}{url}\usepackage{hyperref}
\newcommand{\new}[1]{\textcolor{red}{#1}}
%\usepackage{psfig}
\usepackage{pgfplots}

\newcommand{\future}[1]{\textcolor{red}{#1}}

\newcommand{\hP}{\hat P}
\newcommand{\hp}{\hat p}

\newcommand{\Dk}{\Delta_k}
\newcommand{\Px}{P(x)}
\newcommand{\Qx}{Q(x)}
\newcommand{\Nx}{N_x}

\newcommand{\Py}{P(y)}
\newcommand{\Qy}{Q(y)}
\newcommand{\Pml}{P_{ML}}
\newcommand{\Pmlx}{\Pml(x)}
\newcommand{\Pbeta}{P_{\beta}}
\newcommand{\Pbetax}{\Pbeta(x)}

\newcommand{\dTV}[2]{d_{TV} (#1,#2)}
\newcommand{\dKL}[2]{D(#1||#2)}
\newcommand{\chisq}[2]{\chi^2(#1,#2)}
\newcommand{\eps}{\varepsilon}

\newcommand{\nPepsp}[1]{n^*(#1, \eps)}
\newcommand{\nPeps}{\nPepsp{\cP}}


\newcommand{\sumX}{\sum_{x\in\cX}}

\newcommand{\Bpr}[1]{Bern(#1)}

\newenvironment{problem}[2][Problem]{\begin{trivlist}
\item[\hskip \labelsep {\bfseries #1}\hskip \labelsep {\bfseries #2.}]}{\end{trivlist}}

% Theorem-like environments

\newtheorem{Theorem}{Theorem}
\newtheorem{Theorem*}{Theorem}

\newtheorem{Claim}[Theorem]{Claim}
\newtheorem{Claim*}[Theorem]{Claim}
\newtheorem{Corollary}[Theorem]{Corollary}
\newtheorem{Conjecture}[Theorem]{Conjecture}
\newtheorem{CounterExample*}{$\overline{\hbox{\bf Example}}$}
\newtheorem{Definition}[Theorem]{Definition}
\newtheorem{Example}[Theorem]{Example}
\newtheorem{Example*}[Theorem]{Example}
\newtheorem{Exercise}[Theorem]{Exercise}
\newtheorem{Intuition*}[Theorem]{Intuition}
\newtheorem{Joke*}[Theorem]{Joke}
\newtheorem{Lemma}[Theorem]{Lemma}
\newtheorem{Lemma*}[Theorem]{Lemma}
\newtheorem{Open problem}[Theorem]{Open problem}
\newtheorem{Proposition}[Theorem]{Proposition}
\newtheorem{Property}[Theorem]{Property}
\newtheorem{Question}[Theorem]{Question}
\newtheorem{Question*}[Theorem]{Question}
\newtheorem{Remark}[Theorem]{Remark}
\newtheorem{Result}[Theorem]{Result}
\newtheorem{Fact}[Theorem]{Fact}
\newtheorem{Condition}[Theorem]{Condition}

\newcommand{\ed}{\stackrel{\mathrm{def}}{=}}
%\newcommand{\edef}{\stackrel{\mathrm{def}}{=}}
\def \Paren#1{{\left({#1}\right)}}

\newcommand{\probof}[1]{\Pr\Paren{#1}}
\newcommand{\proboff}[1]{p\Paren{#1}}


\newtheorem{theorem}{Theorem}
\newtheorem{proposition}[theorem]{Proposition}
\newtheorem{corollary}[theorem]{Corollary}
%\newtheorem*{corollary*}{Corollary}
\newtheorem{assumption}[theorem]{Assumption}
\newtheorem{lemma}[theorem]{Lemma}
%\newtheorem*{lemma*}{Lemma}
\newtheorem{conjecture}[theorem]{Conjecture}
\newtheorem{example}[theorem]{Example}
\newtheorem{definition}[theorem]{Definition}
\newtheorem{claim}[theorem]{Claim}

\newcommand{\Xon}{X_1^n}


\newcommand{\prob}{{\rm Pr}}
\newcommand{\Probof}[1]{\prob\left(#1\right)}



\newcommand{\ignore}[1]{}

% Equation formatting

\newcommand{\spreqn}[1]{{\qquad\text{#1}\qquad}}

% Blackboard fonts
\newcommand{\II}{\mathbb{I}} % Added by Theertha on April 16th 2013.
\newcommand{\EE}{\mathbb{E}}
\newcommand{\CC}{\mathbb{C}}
\newcommand{\NN}{\mathbb{N}}
\newcommand{\QQ}{\mathbb{Q}}
\newcommand{\RR}{\mathbb{R}}
\newcommand{\ZZ}{\mathbb{Z}}
\newcommand{\PP}{\mathbb{P}}

% Number sets

\newcommand{\complex}{\CC}
\newcommand{\integers}{\ZZ}
\newcommand{\naturals}{\NN}
\newcommand{\positives}{\PP}
\newcommand{\rationals}{\QQ}
\newcommand{\reals}{\RR}

\newcommand{\realsge}{{\reals_{\ge}}}
\newcommand{\realsp}{\reals^+}
\newcommand{\integersp}{\integers^+}
\newcommand{\integerss}[1]{\integers_{\ge{#1}}}

% boldface

\def \ba     {{\bf a}}
\def \bx     {{\bf x}}
\def \by     {{\bf y}}

\def \bA     {{\bf A}}
\def \bB     {{\bf B}}
\def \bC     {{\bf C}}
\def \bD     {{\bf D}}
\def \bF     {{\bf F}}
\def \bG     {{\bf G}}
\def \bL     {{\bf L}}
\def \bQ     {{\bf Q}}
\def \bR     {{\bf R}}
\def \bS     {{\bf S}}
\def \bT     {{\bf T}}
\def \bX     {{\bf X}}
\def \bY     {{\bf Y}}
\def \bZ     {{\bf Z}}

% caligraphics

\def \cA     {{\cal A}}
\def \cB     {{\cal B}}
\def \cC     {{\cal C}}
\def \cD     {{\cal D}}
\def \cE     {{\cal E}}
\def \cF     {{\cal F}}
\def \cG     {{\cal G}}
\def \cH     {{\cal H}}
\def \cI     {{\cal I}}
\def \cK     {{\cal K}}
\def \cL     {{\cal L}}
\def \cM     {{\cal M}}
\def \cN     {{\cal N}}
\def \cO     {{\cal O}}
\def \cP     {{\cal P}}
\def \cQ     {{\cal Q}}
\def \cR     {{\cal R}}
\def \cS     {{\cal S}}
\def \cT     {{\cal T}}
\def \cU     {{\cal U}}
\def \cV     {{\cal V}}
\def \cW     {{\cal W}}
\def \cX     {{\cal X}}
\def \cY     {{\cal Y}}
\def \cZ     {{\cal Z}}

% vectors

\def \vct#1{{\overline{#1}}}

\def \vcta  {{\vct a}}
\def \vctb  {{\vct b}}
\def \vctq  {{\vct q}}
\def \vcts  {{\vct s}}
\def \vctu  {{\vct u}}
\def \vctv  {{\vct v}}
\def \vctx  {{\vct x}}
\def \vcty  {{\vct y}}
\def \vctz  {{\vct z}}
\def \vctp  {{\vct p}}

\def \vctV  {{\vct V}}
\def \vctX  {{\vct X}}
\def \vctY  {{\vct Y}}
\def \vctZ  {{\vct Z}}

\def \vctbeta  {{\vct\beta}}



% qed's --  Also consider \qedhere

\def \eqed    {\eqno{\qed}}
\def \rqed    {\hbox{}~\hfill~$\qed$}

% sequences

\def \upto  {{,}\ldots{,}}

\def \zn    {0\upto n}
\def \znmo  {0\upto n-1}
\def \znpo  {0\upto n+1}
\def \ztnmo {0\upto 2^n-1}
\def \ok    {1\upto k}
\def \on    {1\upto n}
\def \onmo  {1\upto n-1}
\def \onpo  {1\upto n+1}

% sets

\def \sets#1{{\{#1\}}}
\def \Sets#1{{\left\{#1\right\}}}

\def \set#1#2{{\sets{{#1}\upto{#2}}}}

\def \setpmo   {\sets{\pm 1}}
\def \setmpo   {\{-1{,}1\}}
\def \setzo    {\{0{,}1\}}
\def \setzn    {\{\zn\}}
\def \setznmo  {\{\znmo\}}
\def \setztnmo {\{\ztnmo\}}
\def \setok    {\{\ok\}}
\def \seton    {\{\on\}}
\def \setonmo  {\{\onmo\}}
\def \setzon   {\setzo^n}
\def \setzos   {\setzo^*}


% Set operations



\def\ignore#1{}

 

\title{Assignment Five\\ ECE 4200/5420}
\date{}

\begin{document}
\maketitle 


\begin{itemize}
\item
Provide credit to \textbf{any sources} other than the course staff that helped you solve the problems. This includes \textbf{all students} you talked to regarding the problems. 	
\item
You can look up definitions/basics online (e.g., wikipedia, stack-exchange, etc).
\item
{\bf The due date is 10/18/2021, 23.59.59 ET}. 
\item
Submission rules are the same as previous assignments.
\item
\textbf{Please write your net-id on top of every page. It helps with grading.}
\end{itemize}



\begin{problem}{1. (15 points)}
SVM's obtain \emph{non-linear} decision boundaries by mapping the feature vectors $ X\in \RR^d$ to a possibly high dimensional space via a function $\phi:\RR^d\to\RR^m$, and then finding a linear decision boundary in the new space. 

We also saw that to implement SVM, it suffices to know the kernel function $K( X_i,  X_j)=\phi( X_i)\cdot \phi( X_j)$, without even explicitly specifying the function $\phi$. 

Recall \textbf{Mercer's theorem}. $K$ is a kernel function if and only if for any $n$ vectors, $ X_1\upto  X_n\in \RR^d$, and \textbf{any} real numbers $c_1, \ldots, c_n$, $\sum_{i=1}^n\sum_{j=1}^n c_i c_j K( X_i,  X_j)\ge0$. 

\begin{enumerate}
\item 
Prove the following half of Mercer's theorem (which we showed in class). If $K$ is a kernel then $\sum_{i=1}^n\sum_{j=1}^n c_i c_j K( X_i, X_j)\ge0$. 
\item
Let $d=1$, and $x,y\in \RR$. Is the function $K(x,y)=x+y$ a kernel? 
\item 
Let $d=1$, and $x,y\in\RR$. Is $K(x,y)=xy+1$ a kernel?
\item
Suppose $d=2$, namely the original features are of the form $ X = [ X^1,  X^2]$. Show that $K( X, Y)=(1+\bar X\cdot \bar Y)^2$ is a kernel function. This is called as \textbf{quadratic kernel}.
 
(\textbf{Hint}: Find a $\phi:\RR^2\to \RR^m$ (for some $m$) such that $\phi( X)\cdot \phi( Y) = (1+ X\cdot Y)^2$). 
\item
Consider the training examples $\langle[0, 1], 1\rangle, \langle[1, 2], 1\rangle,\langle[-1, 2], 1\rangle,\langle[0, 11], 1\rangle, \langle[3, 4], -1\rangle, \langle[-3, 4], -1\rangle,$ $ \langle[1 -1], -1\rangle, \langle[-1, -1], -1\rangle$. We have plotted the data points below.
\begin{itemize}
\item
Is the data \textbf{linearly classifiable} in the original 2-d space? 
If yes, please come up with \emph{any} linear decision boundary that separates the data.
If no, please explain why.
\item 
Is the data linearly classifiable in the feature space corresponding to the quadratic kernel. 
If yes, please come up with \emph{any} linear decision boundary that separates the data.
If no, please explain why.
\end{itemize}
\end{enumerate}

\begin{center}
\begin{tikzpicture}[scale=1.5]
\begin{axis}[
    axis lines=middle,
    xmin=-8, xmax=8,
    ymin=-5, ymax=16,
    xtick=, ytick=
]
\addplot [only marks, mark=x, mark options={red}] table {
0 1
1 2
-1 2   
0 11
};
\addplot [only marks, mark=o, mark options={blue}] table {
3 4
-3 4
1 -1
-1 -1
};
\end{axis}
\end{tikzpicture}
\end{center}

\end{problem}

\begin{problem}{2. (10 points)}
The Gaussian kernel (also called Radial Basis Function kernel (RBF)) is:
\[
K( X, Y) = \exp\Paren{-\frac{\| X- Y\|_2^2}{2\sigma^2}},
\]
where $ X, Y$ are feature vectors in $d$ dimensions.
Suppose $d=1$, and $2\sigma^2=1$. 
\begin{enumerate}
\item Design a function $\phi:\RR\to \RR^m$ that corresponds to Gaussian kernel for $d=1$, and $2\sigma^2=1$. 

\textbf{Hint:} Use Taylor series expansion for the exponential function. 
\item What is the value of $m$ you end up with?
\end{enumerate}
\end{problem}

\begin{problem}{3. (10 points)}
Let $f, h_i, 1\leq i\leq n$ be functions from $\mathbb{R}^m\to\mathbb{R}$ (real-valued functions) and let $\bm{\alpha} \in \mathbb{R}^n$. Let $L(z,\bm{\alpha}) = f(z) + \sum\limits_{i=1}^{n} \alpha_i h_i(z)$. In this problem, we will prove that the following two optimization problems are equivalent.

\begin{minipage}{0.45\textwidth}
\begin{equation}\label{eq:constrained_opti}
\begin{aligned}
& \min_{\bm{z}} f(\bm{z}) \\
& \text{s.t. } h_i(\bm{z}) \leq 0, \; i = 1, \ldots, n.
\end{aligned}
\end{equation} 
\end{minipage}%
\hfill
\begin{minipage}{0.45\textwidth}
\begin{equation}\label{eq:minmax}
\min_{\bm{z}} \max_{\bm{\alpha} \geq \bm{0}} L(\bm{z},\bm{\alpha})
\end{equation}
\end{minipage}

Let $(\bm{z}^*,\bm{\alpha}^*)$ be the solution of \eqref{eq:minmax} and let $\bm{z}_p^*$ be the solution of \eqref{eq:constrained_opti}. 
\begin{enumerate}
\item Prove that:
\[
L(\bm{z}^*, \bm{\alpha}^*)  = f(\bm{z}_p^*).
\]
\item \textbf{(complementary slackness).} Let $\bm{\alpha}^*=(\bm{\alpha}^*_1,\ldots,\bm{\alpha}^*_n)$. Show that for all $1\le i\le n$:
\[
\bm{\alpha}^*_i\cdot h_i(\bm{z}^*)=0.
\]
\end{enumerate}

\textbf{Hint}: For $\bm{z}\in\mathbb{R}^m$, let $\bm{\alpha}_{\bm{z}} = \arg \max_{\bm{\alpha} \geq \bm{0}} L(\bm{z},\bm{\alpha})$, which is a function of $\bm{z}$. Then note that \eqref{eq:minmax} is the same as $\min_{\bm{z}} L(\bm{z}, \bm{\alpha}_{\bm{z}})$ Use the fact that for any $\bm{z}$, 
$\bm{\alpha} \geq \bm{0}$,  $L(\bm{z}^*,\bm{\alpha}^*) \geq L(\bm{z}^*,\bm{\alpha})$ and $L(\bm{z}^*, \bm{\alpha}^*) \leq L(\bm{z},\bm{\alpha}_{\bm{z}})$.

You may follow the following steps but it is not required as long as your proof is correct.
\begin{enumerate}
\item Prove that $L(\bm{z}^*, \bm{\alpha}^*) \leq f(\bm{z}_p^*)$
\item Prove that $L(\bm{z}^*, \bm{\alpha}^*) \geq f(\bm{z}_p^*)$
\end{enumerate}
\end{problem}


\begin{problem}{4 (25 points) SVM Classification}
	Please refer to the Jupyter Notebook in the assignment, and complete the coding part in it!
	You can use sklearn SVM package: \url{https://scikit-learn.org/stable/modules/generated/sklearn.svm.SVC.html#sklearn.svm.SVC}
\end{problem}


\end{document}